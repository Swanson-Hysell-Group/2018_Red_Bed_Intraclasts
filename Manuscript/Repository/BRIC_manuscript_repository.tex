\documentclass[11pt,letterpaper]{article}

\usepackage{textcomp,marvosym}
\usepackage{amsmath,amssymb}
\usepackage[left]{lineno}
\usepackage{changepage}
\usepackage{rotating}
\usepackage{natbib}
\usepackage{setspace}
\usepackage{fancyhdr}
\usepackage{graphicx}
\usepackage[aboveskip=1pt,labelfont=bf,labelsep=period,justification=raggedright,singlelinecheck=off]{caption}
%\doublespacing

\raggedright
\textwidth = 6.5 in
\textheight = 8.25 in
\oddsidemargin = 0.0 in
\evensidemargin = 0.0 in
\topmargin = 0.0 in
\headheight = 0.0 in
\headsep = 0.5 in
\parskip = 0.1 in
\parindent = 0.2in

%\pagestyle{myheadings}
%\pagestyle{fancy}
%\fancyhf{}
%\lhead{Swanson-Hysell et al., to be submitted to GEOLOGY}
%\rhead{\thepage}

\begin{document}

\begin{flushleft}
{\Large \textbf{Data repository for ``Primary red bed magnetization revealed by fluvial intraclasts''}}

\end{flushleft}

\section*{Additional study location information}
The study site in the Freda Formation along the Bad River is within the Ashland Syncline (47.3867\textdegree N, 90.6371\textdegree W, WGS84; Fig. \ref{fig:location_figure}A,B). These outcrop exposures along the Bad River are very fresh as the soil-rock interface dates to retreat from the last glacial maximum which is constrained locally to be 13.2 $\pm$ 0.4 thousand years ago based on nearby $^{10}$Be exposure dates \citep{Ullman2015a}. The outcrops have been subsequently been exposed through ongoing river incision (Fig. \ref{fig:location_figure}C). The rocks are very well-preserved for their antiquity. In contrast to localities $\sim$90 km to the east in the White Pine region, there is a lack of mineralization in the underlying Nonesuch Formation in the Ashland Syncline \citep{Stewart2017a}. 

\begin{figure}[!ht]
\noindent\includegraphics[width=\textwidth]{location_figure.pdf}
\caption{\small{Geological maps of the study region highlighting bedrock units associated with the Midcontinent Rift. The study location (47.3867\textdegree N, 90.6371\textdegree W, WGS84) is shown as a red star on the Lake Superior region overview map (A), the zoom-in map of the eastern Ashland sycline (B) and the Bad River for which the geology is overlain on a satellite image (ESRI World Imagery). CHC stands for Copper Harbor Conglomerate. The geologic map data have been modified from \cite{Survey2011a}, \cite{Nicholson2004a}, and \cite{Jirsa2011a}.}}
\label{fig:location_figure}
\end{figure} 

\begin{figure}[!ht]
\noindent\includegraphics[width=\textwidth]{Repository_Image_Figure.png}
\caption{\small{Scanning electron microscopy images and data. Energy-dispersive X-ray spectroscopy (EDS) spectrum are shown for individual labeled grains as are electron backscatter diffraction (EBSD) images. These techniques along with optical reflected light microscopy were used to identify the grains labeled as hematite in Figure 1 of the main text. The EDS map shows the distribution of Fe and Ti for the same view as the backscatter image. This map reveals both micron-scale oxide grains and disseminated iron within clays and the matrix.}}
\label{fig:location_figure}
\end{figure} 

\newpage

\bibliographystyle{gsabull}
\bibliography{../../../../0000_Github/references/allrefs}

\end{document}
