\documentclass[11pt,letterpaper]{article}

\usepackage{textcomp,marvosym}
\usepackage{amsmath,amssymb}
\usepackage[left]{lineno}
\usepackage{changepage}
\usepackage{rotating}
\usepackage{natbib}
\usepackage{setspace}
\usepackage{fancyhdr}
\usepackage{graphicx}
\usepackage[aboveskip=1pt,labelfont=bf,labelsep=period,justification=raggedright,singlelinecheck=off]{caption}
\doublespacing

\raggedright
\textwidth = 6.5 in
\textheight = 8.25 in
\oddsidemargin = 0.0 in
\evensidemargin = 0.0 in
\topmargin = 0.0 in
\headheight = 0.0 in
\headsep = 0.5 in
\parskip = 0.1 in
\parindent = 0.2in

%\pagestyle{myheadings}
%\pagestyle{fancy}
%\fancyhf{}
%\lhead{Swanson-Hysell et al., to be submitted to GEOLOGY}
%\rhead{\thepage}

\begin{document}

\begin{flushleft}
{\Large \textbf{Data repository for ``Primary red bed magnetization revealed by fluvial intraclasts''}}
\\
Nicholas L. Swanson-Hysell\textsuperscript{1},
Luke M. Fairchild\textsuperscript{1},
Sarah P. Slotznick\textsuperscript{1}
\\
\bigskip
\textsuperscript{1} Department of Earth and Planetary Science, University of California, Berkeley, CA, USA
\bigskip

\end{flushleft}

The study site is within the Ashland Syncline where there is a lack of mineralization in contrast with localities $\sim$90 km to the east in the White Pine region \citep{Stewart2017a}. These outcrop exposures along the Bad River are very fresh as the soil-rock interface dates to retreat from the last glacial maximum which is constrained locally to be 13.2 $\pm$ 0.4 thousand years ago based on nearby $^{10}$Be exposure dates \citep{Ullman2015a}. The outcrops have been subsequently been exposed through ongoing river incision. 

\newpage

\bibliographystyle{gsabull}
\bibliography{../../../0000_Github/references/allrefs}

\end{document}